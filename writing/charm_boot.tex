\documentclass{article}
%\usepackage{hyperref}
\usepackage{amsmath}

\usepackage{xcolor}
\usepackage[colorlinks = true,
            linkcolor = blue,
            urlcolor  = blue,
            citecolor = blue,
            anchorcolor = blue]{hyperref}

%\textheight 300mm
%\setlength{\topmargin}{0 in}
%\setlength{\oddsidemargin}{0in}
%\setlength{\evensidemargin}{0in}
%\baselineskip 7mm
%\usepackage{lipsum}
%\usepackage{enumitem}

\usepackage[a4paper, total={6in, 10in}]{geometry}



\begin{document}

\centerline{\LARGE{\textbf{Error propagation in charm baryon spectroscopy}}}
\smallskip
\centerline{Andres Ramirez-Morales and Hugo Garcia-Tecocoatzi}

\section*{Introduction}

Mass model Hamiltonian that determines the mass splitting of the charm baryon states.
\begin{equation}
\label{eqn:mass}
H = H_{\mathrm{h.o.}} + \textcolor{red}{A}\mathbf{S}^{2} + \textcolor{red}{B}\mathbf{S\cdot L} +  \textcolor{red}{E}\mathbf{I}^{2} + \textcolor{red}{G}\mathbf{C_{2}}(\mathrm{SU(3)_{f}})
\end{equation}

\section*{Parameter determination and error propagation (paper version)}

The experimentally observed charm baryon mass states, $\Omega_{C}$,  $\Sigma_{C}$, $\Lambda_{C}$, $\Xi_{C}$, were fitted to the model described in Equation \ref{eqn:mass} to obtain the parameters $\omega, A, B, E, G$. The fitted model parameters minimize the sum of the squared differences between the experimental and model predicted masses (least squares method).\\

The measured baryon masses  reported in Ref. [1-10] come with statistical and systematic uncertainties.  Furthermore, an uncertainty of 10 MeV on the quark system mass is considered.  To account for these uncertainties into our fitted model parameters, we carried out a statistical simulation for error propagation, that is, we randomly sampled the experimental masses, utilizing a Gaussian shape distribution having a mean equal to the central mass value and with a width equal to the squared sum of the uncertainties. We fitted the model parameters using a sampled mass, corresponding to each experimental observed state, and repeat the procedure $10^5$ times. In this manner, we obtained a Gaussian distribution (a formal normality test was performed as a cross-check) for each model parameter and the mass itself; next, we assigned the mean of the distribution as the value of the parameter and used the quantile at 68\% C.L. to extract the uncertainty. The methodology just described is referred to as Monte Carlo bootstrap uncertainty propagation [Aqui citamos al fernandez y los papers que el da].\\

In order to gain better understanding on the different charm baryons, we performed several fits, grouping them according to their quark composition. First, we performed the fit including all the available states,  results are reported in Table \ref{tab:All_mass_paper}.  Next, we performed the fit considering only the $\Omega_{C}$ states,  results are reported in Table \ref{tab:omega_mass_paper}.  Furthermore, we fitted considering only the cascade states, this is shown in Table \ref{tab:cascades_mass_paper}.  The final group combine the $\Lambda_{C}$ and $\Sigma_{C}$ states together, results are shown in Table \ref{tab:sigmaLamb_mass_paper}.

\begin{table}[h!]
\input{./tables/masses_All_paper.tex}
\end{table}

\begin{table}[h!]
\input{./tables/masses_omega_paper.tex}
\end{table}

\begin{table}[h!]
\input{./tables/masses_cascades_paper.tex}
\end{table}

\begin{table}[h!]
\input{./tables/masses_sigmaLamb_paper.tex}
\end{table}


The correlation between the fit parameters could be used to throw light on the mass spectroscopy, Tables \ref{tab:All_corr}-\ref{tab:sigmaLamb_corr}

\begin{table}[h!]
\begin{tabular}{c  c  c  c  c  c}\hline \hline
     &  $K$      &     $A$   &      $B$   &      $E$  & $G$ \\ \hline
 $K$ &     1     &           &            &           &   \\ 
 $A$ & 0.24 &      1    &            &           &   \\ 
 $B$ & 0.43 & 0.46 &      1     &           &   \\ 
 $E$ & 0.1 & 0.1 & -0.03 &      1    &   \\ 
 $G$ & -0.53 & -0.71 & -0.23 & -0.49 & 1 \\ \hline \hline
\end{tabular}

\end{table}

\begin{table}[h!]
\begin{tabular}{c  c  c  c  c  c}\hline \hline
     &  $K$      &     $A$   &      $B$   &      $E$  & $G$ \\ \hline
 $K$ &     1     &           &            &           &   \\ 
 $A$ & nan &      1    &            &           &   \\ 
 $B$ & nan & nan &      1     &           &   \\ 
 $E$ & nan & nan & nan &      1    &   \\ 
 $G$ & nan & nan & nan & nan & 1 \\ \hline \hline
\end{tabular}

\end{table}

\begin{table}[h!]
\begin{tabular}{c  c  c  c  c  c}\hline \hline
     &  $K$      &     $A$   &      $B$   &      $E$  & $G$ \\ \hline
 $K$ &     1     &           &            &           &   \\ 
 $A$ & 0.12 &      1    &            &           &   \\ 
 $B$ & 0.38 & 0.59 &      1     &           &   \\ 
 $E$ & -0.49 & 0.05 & -0.15 &      1    &   \\ 
 $G$ & 0.06 & -0.59 & -0.17 & -0.74 & 1 \\ \hline \hline
\end{tabular}

\end{table}

\begin{table}[h!]
\input{./tables/correlation_sigmaLamb.tex}
\end{table}

\section*{Parameter determination and error propagation (long version)}

\subsection*{Introduction}

Propagation of uncertainty is a building block of the statistical reasoning. The idea of propagation of error using computer simulations, was first proposed by Efron, and is referred to as \textit{bootstrap} methods Ref.[1].  The bootstrap rest in two ideas: \textit{i)} the variability of the data, based on a statistical model, may be estimated reasonably accurately,  and \textit{ii)} this variability may be propagated to express uncertainty about any quantities computed from the data. There are two types of bootstrap methods:

\begin{itemize}
\item\textit{Parametric bootstrap:}
\item\textit{Non-parametric bootstrap:}
\end{itemize}

\subsection{Parameter comparison}
The model parameters calculated using Monte Carlo simulation are compared with the previous parameters presented in Ref.[11].An additional cross-check was carried out, that is, we cast into matrix form the baryon mass model; the matrix equation was solved using \texttt{Octave} framework, and the parameters.  A very good agreement is found between the parameters computed with simulation and the ones using the matrix form. Results are shown in Tables \ref{tab:All_param}-\ref{tab:sigmaLamb_param}



\begin{table}[h!]
\begin{tabular}{c | c  c  c  c  c }\hline \hline
          & $K$        & $A$             & $B$     & $E$        & $G$            \\ \hline
Paper     & 5727.12$\pm x.xx$ & 21.54$\pm 0.37$ & 23.91$\pm 0.31$ & 30.34 $\pm 0.23$ & 54.37$\pm 0.58$ \\ 
Sampled   & 5339.2  $\pm 11.38 $ & 22.26  $\pm 0.46 $ & 21.07  $\pm 0.47 $ & 28.62  $\pm 1.02 $ & 57.16  $\pm 0.36 $ \\ 
L.Algebra & 5339.1  & 22.24  & 21.05  & 28.63 & 57.17  \\ 
\hline\hline
\end{tabular}
\caption{Model pararameters in MeV, for states: $ All $}

\end{table}

\begin{table}[h!]
\begin{tabular}{c | c  c  c  c  c }\hline \hline
          & $K$        & $A$             & $B$     & $E$        & $G$            \\ \hline
Paper     & 5727.12$\pm x.xx$ & 21.54$\pm 0.37$ & 23.91$\pm 0.31$ & 30.34 $\pm 0.23$ & 54.37$\pm 0.58$ \\ 
Sampled   & 5995.1  $\pm 26.74 $ & 26.79  $\pm 0.48 $ & 31.92  $\pm 0.52 $ & 0.0  $\pm 0.0 $ & 49.56  $\pm 0.54 $ \\ 
L.Algebra & 5994.5  & 26.8  & 31.93  & 0.0 & 49.56  \\ 
\hline\hline
\end{tabular}
\caption{Model pararameters in MeV, for states: $ omega $}

\end{table}

\begin{table}[h!]
\begin{tabular}{c | c  c  c  c  c }\hline \hline
          & $K$        & $A$             & $B$     & $E$        & $G$            \\ \hline
Paper     & 5727.12$\pm x.xx$ & 21.54$\pm 0.37$ & 23.91$\pm 0.31$ & 30.34 $\pm 0.23$ & 54.37$\pm 0.58$ \\ 
Sampled   & 5463.8  $\pm 30.43 $ & 20.99  $\pm 0.7 $ & 23.82  $\pm 0.73 $ & 34.87  $\pm 4.75 $ & 56.76  $\pm 1.11 $ \\ 
L.Algebra & 5462.3  & 21.0  & 23.82  & 35.04 & 56.73  \\ 
\hline\hline
\end{tabular}
\caption{Model pararameters in MeV, for states: $ cascades $}

\end{table}

\begin{table}[h!]
\input{./tables/parameters_sigmaLamb.tex}
\end{table}


\subsection{Parameter}

\begin{table}[h!]
\input{./tables/masses_All_note.tex}
\end{table}

\begin{table}[h!]
\input{./tables/masses_omega_note.tex}
\end{table}

\begin{table}[h!]
\input{./tables/masses_cascades_note.tex}
\end{table}

\begin{table}[h!]
\input{./tables/masses_sigmaLamb_note.tex}
\end{table}



\subsection*{Fundamental concepts}
\begin{itemize}
\item Monte Carlo simulation: is a mathematical technique that generates random variables for modeling uncertainty of a given system. Note that MC simulation provides a probabilistic estimate of the uncertainty. MC simulations rely on repeated random sampling to obtain numerical results.
\item Variance refers to the spread of a data set around its mean value, while covariance refers to the measure of the directional relationship between two random variables.
\item Andres

\end{itemize}


\begin{pmatrix}
1 & 2 & 3\\
a & b & c
\end{pmatrix}

\begin{thebibliography}{9}

\bibitem{energy} 
M. Farina,  $et.\,al.$,
\textit{Energy helps accuracy: electroweak precision tests at hadron colliders}.
Phys. Lett. B. \textbf{772}, 210-215 (2017),
\url{https://doi.org/10.1016/j.physletb.2017.06.043}

\bibitem{EDM}
V. Andreev, $et.\,al.$, (ACME Collaboration)
\textit{Improved limit on the electric dipole moment of the electron}.
Nature \textbf{562}, 355-360(2018),
\url{https://www.nature.com/articles/s41586-018-0599-8}

\bibitem{belle1}
K. Abe, $et.\,al.$, (BELLE Collaboration)
\textit{Observation of Large CP Violation in the Neutral B Meson System}.
Phys. Rev. Lett. \textbf{87}, 091802(2001),
\url{https://journals.aps.org/prl/abstract/10.1103/PhysRevLett.87.091802}

\bibitem{belle2}
I. Adachi, $et.\,al.$, (BELLE Collaboration)
\textit{Precise Measurement of the $CP$ Violation Parameter $\mathrm{sin}2{\ensuremath{\phi}}_{1}$ in ${B}^{0}\ensuremath{\rightarrow}(c\overline{c}){K}^{0}$ Decays}.
Nature \textbf{108}, 171802(2012),
\url{https://journals.aps.org/prl/abstract/10.1103/PhysRevLett.108.171802}

\bibitem{npdgamma}
D. Blyth, $et.\,al.$, (NPDGamma Collaboration),
\textit{First Observation of $P$-odd $\ensuremath{\gamma}$ Asymmetry in Polarized Neutron Capture on Hydrogen}.
Phys. Rev. Lett. \textbf{121},  242002 (2018)
\url{https://link.aps.org/doi/10.1103/PhysRevLett.121.242002} 
 
\bibitem{QWSE}
A. Ramirez-Morales, $et.\,al.$,
\textit{Improvement of the quantum confined Stark effect characteristics by means of energy band profile modulation: The case of Gaussian quantum wells},
Journal of Applied Physics  \textbf{110}, 103715 (2011)
\url{https://doi.org/10.1063/1.3662907}

\bibitem{n3he}
M. Gericke,
\textit{The n3He experiment: Hadronic parity violation in cold neutron capture on 3He},
Proceedings of Science \textbf{253}, (2016)
\url{https://doi.org/10.22323/1.253.0102}

\bibitem{highZ}
M. Aaboud $et\,al.$ (ATLAS Collaboration),
\textit{Measurement of the double-differential high-mass Drell-Yan cross section in pp collisions at $\sqrt{s}=8$ TeV with the ATLAS detector},
J. High Energ. Phys.  \textbf{2016:9} (2016)
\url{https://doi.org/10.1007/JHEP08(2016)009}

\bibitem{primes}
M. Aaboud $et\,al.$ (ATLAS Collaboration),
\textit{Combination of searches for heavy resonances decaying into bosonic and leptonic final states using 36 fb$^{-1}$ of proton-proton collision data at $\sqrt{s}=13$ TeV with the ATLAS detector},
Phys. Rev. D \textbf{98}, 052008 (2018)
\url{https://journals.aps.org/prd/abstract/10.1103/PhysRevD.98.052008}

\bibitem{stat}
B. Laforge and L. Schoeffel,
\textit{Elements of statistical methods in high-energy physics analyses},
Nuclear Instruments and Methods in Physics Research Section A \textbf{394} 115-120 (1997)
\url{https://doi.org/10.1016/S0168-9002(97)00649-9}

\bibitem{uncBelle}
J. Charles $et al.$,
\textit{Modeling theoretical uncertainties in phenomenological analyses for particle physics},
The European Physical Journal C, \textbf{214}, 1434-6052(2017)
\url{https://doi.org/10.1140/epjc/s10052-017-4767-z}

\bibitem{ewSys}
P. Langacker and M. Luo,
\textit{Implications of precision electroweak experiments for $m_{t}$, $\rho_0$, $\sin^2\theta_W$, and grand unification},
Phys. Rev. D \textbf{44}, 3 (1991)
\url{https://doi.org/10.1103/PhysRevD.44.817}


\bibitem{NN1}
D. Guest, $et.\,al.$,
\textit{Deep Learning and Its Application to LHC Physics},
Annual Review of Nuclear and Particle Science  \textbf{68}, 161-181(2018)
\url{https://doi.org/10.1146/annurev-nucl-101917-021019}

\bibitem{NN2}
K. Albertsson, $et.\,al.$,
\textit{Machine Learning in High Energy Physics Community White Paper},
\url{https://arxiv.org/abs/1807.02876}


\bibitem{machine}
A. Radovic, $et.\,al.$,
\textit{Machine learning at the energy and intensity frontiers of particle physics},
Nature \textbf{560}, 41-48(2018)
\url{https://doi.org/10.1038/s41586-018-0361-2}

\bibitem{vector}
A. Bevan,  $et.\,al.$,
\textit{Support Vector Machines and generalisation in HEP},
Journal of Physics: Conference Series, \textbf{898}, 072021 (2017)
\url{https://doi.org/10.1088\%2F1742-6596\%2F898\%2F7\%2F072021}


\bibitem{keck}
T. Keck,  $et.\,al.$,
\textit{The Full Event Interpretation},
Computing and Software for Big Science, \textbf{6}, 2510-2044(2019)
\url{https://doi.org/10.1007/s41781-019-0021-8}


\bibitem{belleII}
B.A. Shwartz $et al.$, (Belle II collaboration)
\textit{The Belle II Experiment},
Nuclear and Particle Physics Proceedings \textbf{260} 233-237 (2015)
\url{https://doi.org/10.1016/j.nuclphysbps.2015.02.049}


\bibitem{beRes1}
M. Berger et at., (Belle collaboration)
\textit{Measurement of the decays ${\mathrm{\ensuremath{\Lambda}}}_{c}\ensuremath{\rightarrow}\mathrm{\ensuremath{\Sigma}}\ensuremath{\pi}\ensuremath{\pi}$ at Belle}
Phys. Rev. D \textbf{98}, 112006 (2018)
\url{https://journals.aps.org/prd/abstract/10.1103/PhysRevD.98.112006}


\bibitem{beRes2}
B.G. Fulsom et at., (Belle collaboration)
\textit{Observation of $\mathrm{\ensuremath{\Upsilon}}(2S)\ensuremath{\rightarrow}\ensuremath{\gamma}{\ensuremath{\eta}}_{b}(1S)$ Decay}
Phys. Rev. Lett. \textbf{121}, 232001 (2018)
\url{https://journals.aps.org/prl/abstract/10.1103/PhysRevLett.121.232001}

\bibitem{lattice}
Zachary S. Brown, $et. al.$,
\textit{Charmed bottom baryon spectroscopy from lattice QCD}
Phys. Rev. D \textbf{90}, 094507 (2014)
\url{https://journals.aps.org/prd/abstract/10.1103/PhysRevD.90.094507}

\bibitem{lepton}
Simone Bifani $et. al.$ 
\textit{Review of lepton universality tests in B decays}
J. Phys. G: Nucl. Part. Phys. \textbf{46} 023001 (2019)
\url{https://iopscience.iop.org/article/10.1088/1361-6471/aaf5de}

\bibitem{lv1}
A. Gouvea and P. Vogel,
\textit{Lepton Flavor and Number Conservation, and Physics Beyond the Standard Model}
\url{https://arxiv.org/pdf/1303.4097.pdf}

\bibitem{lv2}
A. Abdesselam $et al.$,(Belle collaboration)
\item{Test of lepton flavor universality in $B\rightarrow K^{*}l^{+}l^{-}$ decays at Belle}
\url{https://arxiv.org/abs/1904.02440}




\end{thebibliography}

\end{document}



%\bibitem{atlasM} 
%A. Yamamoto, $et.\,al.$, 
%\textit{The ATLAS central solenoid}.
%Nuclear Instruments and Methods in Physics Research Section A \textbf{584}, 53-74 (2008)
%\url{https://doi.org/10.1016/j.nima.2007.09.047}

%\bibitem{muSys}
%K. Melnikov,
%\textit{On the theoretical uncertainties in the muon anomalous magnetic moment},
%International Journal of Modern Physics A, \textbf{16}, No. 28 (2001)
%\url{https://doi.org/10.1142/S0217751X01005602}
